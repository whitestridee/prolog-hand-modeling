\chapter{Технологический раздел}


\section{Средства реализации}
\hspace{0.6cm}Для реализации программы были следующие языки программирования:
\begin{itemize}
	\item Python (v.3.8\cite{web:py}) для написания интерфейса программы и отрисовки рук. Python является простым в использовании средством для выполнения небольших задач, таких как чтение и запись, отрисовка оконного интерфейса;
	\item Prolog (SWI-Prolog\cite{web:prolog}) для написания функций проверки точек на корректность.
\end{itemize}

\section{Сборка и запуск проекта}
\hspace{0.6cm}Для сборки проекта используется встроенная система сборки языка Python.
 

\section{Описание структуры базы знаний}
\hspace{0.6cm}Каждый палец (кроме большого) определяется 4 точками (3 точки для большого), следовательно необходимо проверять 3 различных угла при сгибе пальцев (2 для большого), а также угол отклонения пальца при отведении пальца. Таким образом, каждому пальцу соответствует 4 типа проверок.
\begin{lstlisting}[caption=Знания о типах проверок каждого пальца, label=list:finger_check]
%finger_motion_type(FingerType, AbductionType, Flexion1, Flexion2, Flexion3).

finger_motion_type(thumb, bpprived, bppsgib1, bppsgib2, bppsgib2).
finger_motion_type(index, oprived, o2sgib1, o2sgib2, o2sgib3).
finger_motion_type(middle, oprived, o3sgib1, o3sgib2, o3sgib3).
finger_motion_type(ring, oprived, o4sgib1, o4sgib2, o4sgib3).
finger_motion_type(little, oprived, o5sgib1, o5sgib2, o5sgib3).
\end{lstlisting}
Каждому из типу проверок соответствуют диапазоны углов, которые допустимые при том или ином движении пальца.
\begin{lstlisting}[caption=Знания об амплитудах углов, label=list:angle_limits]
%angle_type_limits(Finger, MinAngle, MaxAngle)

angle_type_limits(bpabc, -80, 80).
angle_type_limits(bpbcd, -50, 50).
angle_type_limits(bpcde, -90, 90).
angle_type_limits(oabc, -80, 80).
angle_type_limits(obcd, -100, 100).
angle_type_limits(ocde, -90, 90).
angle_type_limits(between, -30, 30).

angle_type_limits(bpprived, -50, 50).
angle_type_limits(oprived, -60, 60).
angle_type_limits(bppsgib1, -50, 50).
angle_type_limits(bppsgib2, -100, 80).

angle_type_limits(o2sgib1, -120, 90).
angle_type_limits(o2sgib2, -100, 100).
angle_type_limits(o2sgib3, -100, 100).

angle_type_limits(o3sgib1, -120, 90).
angle_type_limits(o3sgib2, -100, 100).
angle_type_limits(o3sgib3, -80, 80).

angle_type_limits(o4sgib1, -120, 90).
angle_type_limits(o4sgib2, -100, 100).
angle_type_limits(o4sgib3, -80, 80).

angle_type_limits(o5sgib1, -120, 90).
angle_type_limits(o5sgib2, -100, 100).
angle_type_limits(o5sgib3, -80, 80).

angle_type_limits(bppz, -100, 100).
\end{lstlisting}
Также каждый из типов проверок отвечает за конкретную ось пространства, по которой проводится проверка.
\begin{lstlisting}[caption=Знания об осях типов проверок, label=list:angle_det_type]
%angle_det_type(Type, Axis)

angle_det_type(bpabc, all).
angle_det_type(bpbcd, all).
angle_det_type(bpcde, all).
angle_det_type(oabc, all).
angle_det_type(obcd, all).
angle_det_type(ocde, all).
angle_det_type(between, all).

angle_det_type(bpprived, x).
angle_det_type(oprived, x).
angle_det_type(bppsgib1, y).
angle_det_type(bppsgib2, y).

angle_det_type(o2sgib1, x).
angle_det_type(o2sgib2, x).
angle_det_type(o2sgib3, x).

angle_det_type(o3sgib1, x).
angle_det_type(o3sgib2, x).
angle_det_type(o3sgib3, x).

angle_det_type(o4sgib1, x).
angle_det_type(o4sgib2, x).
angle_det_type(o4sgib3, x).

angle_det_type(o5sgib1, x).
angle_det_type(o5sgib2, x).
angle_det_type(o5sgib3, x).

angle_det_type(bppz, z).
\end{lstlisting}
Попадание угла в диапазон определяется процедурой, которая для этого использует знание о рассматриваемом виде соединения (который устанавливается исходя из знания о рассматриваемом пальце).
\begin{lstlisting}[caption=Проверка попадания угла в диапазон, label=list:valid_angle]
%valid_angle - check if angle is valid for finger
valid_angle(Type, Angle):-
	angle_type_limits(Type, MinAngle, MaxAngle),
	MinAngle =< Angle, Angle =< MaxAngle.
\end{lstlisting}
Программой на Prolog для описания руки используются структуры Рука, Палец и Точка.
\begin{lstlisting}[caption=Структуры, label=list:structures]
point(X, Y, Z).

hand(
	finger(little, P0, P1, P2, P3),		%5|finger V
	finger(ring, P4, P5, P6, P7),		%4|finger IV
	finger(middle, P8, P9, P10, P11),	%3|finger III
	finger(index, P12, P13, P14, P15),	%2|finger II
	finger(thumb, P16, P17, P18),		%1|finger I
	P19, P20							
).	
\end{lstlisting}
Каждой структуре соответствует своя процедура проверки корректности точек.
\begin{lstlisting}[caption=Процедура проверки корректности точек руки, label=list:validate_hand]
validate_hand(hand:hand(Finger5, Finger4, Finger3, Finger2, Finger1, P19, Wrist)):-
	validate_finger(Finger5, P19, Wrist),
	validate_finger(Finger4, P19, Wrist),
	validate_finger(Finger3, P19, Wrist),
	validate_finger(Finger2, P19, Wrist),
	validate_finger(Finger1, P19, Wrist).
\end{lstlisting}
Процедура проверки корректности точек пальца состоит из двух правил: одно для большого пальца, второе - для остальных пальцев.
\begin{lstlisting}[caption=Процедура проверки корректности точек пальца, label=list:validate_finger]
validate_finger(finger(thumb, P1, P2, P3), P19, Wrist):-
	finger_motion_type(thumb, Abduction, Flex1, Flex2, _),
	validate_points(bpabc, P1, P2, P3),
	validate_points(Abduction, P3, P2, Wrist),
	validate_points(Flex1, P1, P2, P3),
	validate_points(Flex2, P1, P2, Wrist),
	validate_points(bppz, P1, P2, P3).
	
validate_finger(finger(Finger, P1, P2, P3, P4), P19, Wrist):-
	not(Finger == thumb),
	finger_motion_type(Finger, Abduction, Flex1, Flex2, Flex3),
	validate_points(oabc, P1, P2, P3),
	validate_points(obcd, P2, P3, P4),
	validate_points(Abduction, P4, P2, Wrist),
	validate_points(Flex1, P2, P1, P3),
	validate_points(Flex2, P4, P2, P3),
	validate_points(Flex3, P4, P3, Wrist),
	validate_points(bppz, P1, P2, P3).
\end{lstlisting}

\begin{lstlisting}[caption=Процедура проверки угла между точками на корректность, label=list:validate_angle]
validate_angle(Type, Point1, Point2, Point3) :-
	hand:angle_det_type(Type, Axis),
	get_angle(Axis, Point1, Point2, Point3, Angle),
	write_files:write_angle(Type, Angle),
	hand:valid_angle(Type, Angle).
\end{lstlisting}

\section{Взаимодействие с Python}


\section{Отрисовка руки}

