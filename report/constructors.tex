\chapter{Конструкторский раздел}


\section{Углы Эйлера}
\hspace{0.6cm} Углы Эйлера определяют три поворота системы, которые позволяют привести любое положение системы к текущему. Обозначим начальную систему координат как (x, y, z) , конечную как (X,Y,Z). Пересечение координатных плоскостей xy и XY называется линией узлов N.
\begin{itemize}
	\item Угол a между осью x и линией узлов — угол прецессии.
	\item Угол b между осями z и Z — угол нутации.
	\item Угол y между линией узлов и осью  X— угол собственного вращения.
\end{itemize}

\hspace{0.6cm} Повороты системы на эти углы называются прецессия, нутация и поворот на собственный угол (вращение). Такие повороты некоммутативны и конечное положение системы зависит от порядка, в котором совершаются повороты. В случае углов Эйлера производится серия из трёх поворотов:
\begin{enumerate}
	\item На угол a вокруг оси z. При этом ось x переходит в N.
	\item На угол b вокруг оси N. При этом ось z переходит в Z.
	\item На угол y вокруг оси Z. При этом ось N переходит в X.
\end{enumerate}

\hspace{0.6cm} Иногда такую последовательность называют 3,1,3 (или Z,X,Z), но такое обозначение может приводить к двусмыслице.
\hspace{0.6cm} Для вычисления этих углов используются векторы. Генеральная идея если у нас есть 3 точки P1(X1, Y1, Z1),  P2(X2, Y2, Z2),  P3(X3, Y3, Z3) состоит в том что бы найти 2 вектора (Вектор P1P2 и Вектор P2P3) что бы найти угол между ними. Чтобы найти эти 2 вектора мы воспользуемся формулой

\begin{equation} 
\displaystyle AB = (X2 – X1, Y2 – Y1, Z2 – Z1) и BC = (X3 – X2, Y3 – Y2, Z3 – Z2) 
\end{equation}

\hspace{0.6cm} Так как существует формула:

\begin{equation} 
\displaystyle  AB * BC = ||AB|| * ||BC|| * \cos(\theta),
\end{equation}

\hspace{0.6cm} где \theta – угол между этими векторами, то из этой формулы мы получаем этот угол помощу \arccos. Длину вектора находим по формуле:

\begin{equation} 
\displaystyle  ||AB|| = \sqrt{(X*2 + Y*2 + Z*2)}.
\end{equation}

\hspace{0.6cm} Конечная формула получается:

\begin{equation} 
\displaystyle  \theta = \arccos(\frac{AB * BC}{||AB|| * ||BC||}).
\end{equation}

\hspace{0.6cm} Если мы хотим найти угол по оси X или оси Y то мы просто не включаем эти координаты у уровнение, т.е если хотим например найти по оси X то вектор P1P2 можем вычислить вот так:
\begin{equation} 
\displaystyle  P1P2 = (Y2 - Y1, Z2 - Z1).
\end{equation} 
	

\section{Структура проверок положения точек}
\hspace{0.6cm}Входные данные в программу представляют собой 42 точки в трехмерном пространстве. Из 42 точек, 21 определяет одну кисть.

\hspace{0.6cm}Проверка кисти на правильность осуществляется с помощью группы проверок отдельных пальцев, а также точек расположенных непосредственно на ладони. Для реализации проверок лучше использовать язык программирования Prolog, поскольку это мощный инструмент именно для работы с логическими конструкциями.

\hspace{0.6cm}Для удобства представления входных данных их можно разделить на структуры. 
\begin{figure}[ht!]
	\centering
	\includegraphics[scale=0.5]{Kist.jpg}
	\caption{Нумерация точек на кисти}
	\label{fig:hands}
\end{figure}
\hspace{0.6cm}Возьмем точки 0, 1, 2, 3. Вместе они составляют мизинец на руке. В соответствии с этим можно создать структуру мизинца. Схожим образом можно объединить оставшиеся точки в безымянный, средний, указательный и большой пальцы. Останутся только точки 19, 20 на левой руке и 40, 41 на правой. Эти точки являются ключевыми для своих кистей соотвественно.

\hspace{0.6cm}Далее полученные структуры пальцев мы можем объединить в структуру руки. Каждая рука будет состоять из пальцев и оставшимся двум точкам соответственно для левой и правой кисти.

\hspace{0.6cm}Данные преобразования необходимо провести для упрощения понимания структуры кода при его чтении, а также облегчения работы при написании процедур проверок.

\hspace{0.6cm}Сами проверки в своей основе опираются на проверку углов между определенными точками в пальце.  

\section{Визуализация руки}

