\chapter{Конструкторский раздел}


\section{Углы Эйлера}
\hspace{0.6cm}Углы Эйлера определяют три поворота системы, которые позволяют привести любое положение системы к текущему. Обозначим начальную систему координат как (x, y, z) , конечную как (X,Y,Z). Пересечение координатных плоскостей xy и XY называется линией узлов N.
\begin{itemize}
	\item Угол a между осью x и линией узлов — угол прецессии.
	\item Угол b между осями z и Z — угол нутации.
	\item Угол y между линией узлов и осью  X— угол собственного вращения.
\end{itemize}

\hspace{0.6cm}Повороты системы на эти углы называются прецессия, нутация и поворот на собственный угол (вращение). Такие повороты некоммутативны и конечное положение системы зависит от порядка, в котором совершаются повороты. В случае углов Эйлера производится серия из трёх поворотов:
\begin{enumerate}
	\item На угол a вокруг оси z. При этом ось x переходит в N.
	\item На угол b вокруг оси N. При этом ось z переходит в Z.
	\item 	На угол y вокруг оси Z. При этом ось N переходит в X.
\end{enumerate}
Иногда такую последовательность называют 3,1,3 (или Z,X,Z), но такое обозначение может приводить к двусмыслице.


\section{Структура проверок положения точек}


\section{Визуализация руки}

